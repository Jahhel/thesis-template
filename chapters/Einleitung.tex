\chapter{Einleitung}
Mit der zunehmenden Formalisierung von Cybersicherheit, offenkundigen Kriegen innerhalb von Europa und dem beständigen Anstieg von Malwareautorenaktivität zeigt sich auch Bewegung in der Rechtsprechung. So werden mit dem Digital Operation Resiliance Act  nun auch Threat-led Penetration Tests durchgeführt \cite{ab_2024_tlpt} die in vielen Bereichen dem ähneln, was bei SAP die Red-Teams übernehmen. Die Aufgabe von diesartigen Assignments ist es, reale Angreifer im operativen Ablauf zu simulieren und aus deren Erfolgen und Misserfolgen Schlüsse zu ziehen, um ihre eigene Cybersicherheit und die Resilienz gegen (spezifische) Angreifertypen zu stärken.
Neben der grundlegenden Aufgabe von Kompromittierung von Systemen, wie sie in klassischen Penetrationstests vorkommen, stehen Red Teams vor der Aufgabe, ihre Angriffe vor dem Blue-Team (den Verteidigern der Infrastruktur) zu verbergen, um genauso unentdeckt zu bleiben, wie es bei echten Hackerangriffen sein würde.

Naturgemäß bedienen sich dabei Red-Teamer dabei Methodiken und Technologien, wie sie auch von Hackern benutzt werden, um ihre Ziele zu erreichen. Selbstverständlich ist auch, dass böswillige Akteure ihre Ressourcen nicht bereitwillig offenlegen, was es nötig macht, selbst Zeit in die Entwicklung von neuen Technologien zu stecken, um Lücken in bisherigen Verteidigungslinien ausnutzen zu können.

 Eine erste dieser Verteidigungslinien, die fast 90\% \cite{dey_2024_antivirus} von Unternehmen benutzt wird, ist die sogenannte Antivirus-Software (AV), die es von verschiedenen Herstellern gibt. Ihre Aufgabe ist die Abwehr von bereits bekannter Schadsoftware indem sie diese erkennt, isoliert und an der Ausführung ihrer eigentlichen Aufgabe hindert.

Aus diesem Grund ist die Forschung in diesem Bereich stetig gewachsen und über die Jahre ist auch der Forschungsbereich der Malware-Dektection und -Vermeidung aufgekommen, mit der besagten AV-Software, von Malware umgangen werden kann. Hiermit schließt sich der Kreis, denn neben dem bereits bekannten Wettrüsten zwischen AV-Herstellern und Malwareautoren, steigen nun als dritte Partei noch Red-Teams in diese Auseinandersetzung ein, die versuchen, die gleichen Schwachstellen wie böswillige Akteure zu finden, bevor sie von diesen ausgenutzt werden können. Dies erfordert natürlich eigene Forschung in dem Feld, um einen möglichst großen Bedrohungshorizont abdecken zu können.

Eine dieser Möglichkeiten, Malware zu umgehen, die von böswilligen Akteuren stark in Anspruch genommen wird, ist die Erstellung von Malware Varianten, die von AV-Software mit nur geringer Genauigkeit erkannt werden kann. So seien zwischen 50\%\cite{cesare_2013_malwisean, baterdene_2017_packer} und 80\% \cite{kang_2020_a} der Malware im Umlauf einfach obfuskierte, gepackte oder sonstige modifizierte Malware, die schon lange bekannt ist. Dieser Prozess wird in der Wissenschaft auch verwendet, um bessere AV Klassifikatoren zu erstellen\cite{murali_2023_evolving}.

Um diesen Prozess mit möglichst wenigen Ressourcen für Unternehmen und deren Red Teams zu replizieren, benötigt es also Automatisierung, die verschiedene Malware-Varianten erzeugen kann, die von AV-Software nicht erkannt wird. Diese Arbeit soll damit der Frage auf den Grund gehen, inwieweit genetisch optimierte Obfuskatorkaskaden für diese Aufgabe in den Constraints von Red Teams diesen Anspruch füllen können.