\section{Fragestellung}
\label{Sec:Fragestellung}
Aus diesem Grund stellt sich die Frage, ob sich genetische Algorithmen zur Optimierung von Obfuskatorkaskaden unter den Kriterien, die für praktische Red Team Assignments relevant sind, als Werkzeug nutzen lassen. Diese Frage speist sich zum einen aus der ungeheuren Prävalenz von Obfuskatoren und Packern in der Bedrohungslandschaft \cite{alkhateeb_2023_a} und auf der anderen Seite den zeitlichen und zum Teil auch ressourcenbedingten Begrenzungen, die ein Red Team einhalten muss (\ref{Methode:Kriterien}). 

Durch Obfuskation steigt zum einen die Größe der Datei und die Entropie innerhalb der Datei, was beides Faktoren sein können, die von AV-Software erkannt und betrachtet werden können.
