\section{Gütekriterien und Definition}
\label{Methode:Kriterien}
Eine Obfuscation ist dann gut, wenn sie von AV nicht erkannt wird, ausführbar ist und noch ihren Zweck erfüllt. Eine langsame Erkennung durch AV wird ihr positiv angerechnet. Ein Durchlauf eines Genetischen Algorithmus ist gut, wenn er eine Anzahl von guten Obfuskationen > 0 erzeugt und eine geringe Gesamtlaufzeit hat.
Für die praktische Verwendung muss ein solches Tool möglichst autonom und erweiterbar sein. Des Weiteren ist es wichtig, dass es möglich ist, weitere Obfuskationsmöglichkeiten und Schritte einzubauen, um mit der Weiterentwicklung von Malwaretools Schritt halten zu können. Eine ebenso wichtige Anforderung ist die Möglichkeit zur Erweiterung anhand von verschiedenen AV Scannern, um verschiedene Workstations, die als Angriffsziel in Frage kommen, simulieren zu können.
Ein essentieller Schritt ist vor allem anderen aber die Zeitkomponente, da nicht detektierbare Malware für Angriffsszenarien schnell zur Verfügung stehen soll und deshalb weniger als 12 Stunden benötigen muss.