\section{Malware Erkennung}
\label{Sec:Malwaredetektion}
Malware wird komplexer \cite{pascalmaniriho_2023_a}  und damit wird die Herausforderung der Erkennung von Malware ebenso komplexer. Heute geht man soweit, dass Malwareerkennung ein NP Vollständiges Problem ist \cite{ aslan_2020_a,pascalmaniriho_2023_a} und keine Malware Software alle Varianten beliebiger Malwareklassen und Familien erkennen kann \cite{aboaoja_2022_malware}, da neue Malware zunehmend Technologien nutz, um Malwareerkennung und Antivirenprogrammen zu entgehen (Verweis auf Section Malware Erkennung Umgehnung). Angefangen bei Methoden wie Obfuskierung, Packing, dynamischem Malware Verhalten und Fileless Malware \cite{aslan_2020_a}. 

Durch die große Verbreitung (71\% Windows 10, 23\% Windows 11 \cite{Klotz, A}) von Windows als Betriebssystem, wurde diese auch vermehrt ein Ziel von Angreifern, sodass die Entwicklung von Windowsmalware weit fortgeschritten ist und die größte Verbreitung hat \cite{aslan_2020_a}. Diese Entwicklung lässt den Schluss zu, dass auch Angreifer nach einer Maximierung ihrer Effizienz streben und den Großteil ihrer Zeit auf weit verbreitete Tools verwenden werden, sodass die häufigste AV Lösung diese sein wird, bei der die meisten Exploits gefunden werden können. Erste Schritte um mit diesen Problemen umzugehen waren bereits die Impfung von AV Software \cite{murali_2023_evolving} und das Training von ML Klassifikatoren mittels unrealistischer, wissenschaftlich generierter Malware Varianten \cite{dyrmishi_2023_on}.

\subsection{Analyse}
Der Erkennung von Malware liegt die Analyse von Dateien zugrunde, welche Faktoren wie den Datentyp, den Byte oder Optcode von Dateien in Betracht zieht und sich in dynamische, statische und Hybride Analysen aufteilen lässt \cite{aboaoja_2023_a}. Aus den daraus gewonnenen Eigenschaften wird dann eine Klassifikation von Dateien abgeleitet.
\subsubsection{Statische Analyse}
\label{analyse:statisch}
Statische Analysen operieren auf dem File ohne dieses File auszuführen. So können sie den Sourcecode und API-Calls in Betracht ziehen, Signaturen über Source- und Optcodes bilden, oder Signaturen über Abstraktionen, wie zum Beispiel die Abfolge von API Calls\cite{pascalmaniriho_2023_apimaldetect} bilden. Aber auch die Kompressionsrate oder die Entropie einer Datei, sowie die Header eines PE Files können hier einbezogen werden \cite{aboaoja_2023_a, aslan_2020_a}. 

\subsubsection{Dynamische Analyse}
\label{analyse:dynamisch}
Die dynamische Analyse basiert darauf, dass eine potentielle Malware in einer kontrollierten Umgebung, wie einer Sandbox, VM oder emulierten Maschiene ausgeführt und für eine begrenzte Zeit in ihrer Ausführung beobachtet wird\cite{aboaoja_2023_a}. Dabei können Memoryimages während der Laufzeit gemacht werden und Ausführungspfade mitsamt ihrer Edgcases abgedeckt werden, um Verhaltnesdaten während der Ausführung abzubilden. Der wichtige Punkt hieran ist, dass Obfuskatoren am Verhalten des ursprünglichen Programmes nichts ändern, sondern nur Verhalten hinzufügen \cite{alkhateeb_2023_a}. 

\subsubsection{Hybride Analysen}
\label{analyse:hybrid}
Hybride Analysen beinhalten Bestandteile aus \ref{analyse:dynamisch} und \ref{analyse:statisch}. Häufig sind sie in Tools zusammengefügt so wie in \cite{CAPE2}, die automatisierte Prozesse zur Verfügung stellen, um ein Automated Malware Evaluation bereitzustellen, die von einem gegebenen File eine Liste von Attributen und Eigenschaften dieser Datei zur Verfügung stellt. Dieser Bereich ist 

\subsection{Detektion}
Aus den generierten Features dieser Analyse setzen dann die sogenannten Malware Klassifikatoren an; diese teilen Dateien entweder in Gutartig (Benign) oder Bösartig (Malware) ein, oder gehen einen Schritt weiter und treffen Unterscheidungen nach Malwarefamilie und Klasse (z.B. Ransomware, Spyware, Wurm, ...).

\subsubsection{Klassische Klassifikation}
Klassischwerweise spaltet sich die Analyse in die vier Bereiche auf: 
\begin{itemize}
    \item Signatur-basiert
    \item Heuristik-basiert
    \item Verhaltens-basiert
    \item KI-basiert
\end{itemize}

Die \textit{signaturbasierten} Verfahren verlassen sich hauptsächlich auf eine aktuelle Datenbank von Signaturen von bekannter Malware. Diese Signaturen sind nicht notwendigerweise nur über das gesamte File sondern können auch nach bestimmten Sequenzen von API-Aufrufen oder Teilsignaturen beinhalten, um Malware zu erkennen. Dieser Ansatz funktioniert sehr effizient und verlässlich für bekannte Malware, die bereits im System vorhanden ist und erzeugt wenig bis keine False-Positives. Für Endanwender im privaten Kontext reichen diese häufig aus, da sie eine gute Base Level an Protection anbieten. (QUELLE!). 
\textit{Verhaltensbasierte} Klassifikationen stützten sich zu einem großen Teil auf die Auswertungen von dynamischen Analysen und ziehen Dinge in Betracht, wie die Ausführungsgraphen von einer Datei, den gemachten API Calls und den Zugriffen, die dieses Programm macht. Hierbei kann es schneller mal zu False Positives oder Negatives kommen, da eine starke Trennung von Benign und Malware dadurch erschwert wird, dass Malware Autoren ihre Programme der Struktur von Bengingware nachempfinden wollen. ((Quelle)
\textit{Heuristikbasierte} Verfahren, wie \cite{yara}, sind eine Kombination von verschiedenen Tools und geben die Möglichkeit vor spezifische Regeln zu definieren. So kann beispielsweise eine bestimmte Stringfolge in einer Malware in Kombination mit einem Call an eine IP Adresse, deren Location außerhalb des eigenen Landes liegt, als Malware klassifiziert werden. Diese Tools sind relativ weit benutzt, da sie einen hohen Aufwand mit sich bringen, sie zu definieren und zu konfigurieren, ohne die Gewissheit zu haben, inwieweit dies erfolgreich ist und damit häufig in Professionellen Kontexten verwendung findet. ((QUELLEN!)). Hierbei können beispielsweise sehr spezifische Angriffe reduziert oder gänzlich ausgeschlossen werden: So hat sich String erkennung gegen HTML-Tags und JavaScript basierte Angriffe als Erfolgreich erwiesen, DNS Basierte Regeln als Abwehr für Botnets etabliert. Ein anfängliches Thema ist die automatisierte Erstellung von Heuristiken mittels Machinelearnings. (QUELLE)

\textit{KI-Basierte} Malware Klassifikation ist in den letzten Jahren der Fokus der Wissenschaft geworden (QUELLE) und verspricht eine Möglichkeit zur Erkennung von unbekannten Malware Varianten anhand von Kombinationen aus den Daten der Analyse. Ein Nachteil und Schwierigkeit bei diesem Ansatz ist häufig eine mangelnde Erklärbarkeit und Einschätzung von Zuverlässigkeiten. Dies ist einer der Gründe, warum \cite{jiang_2024_benchmfc} Datensets für ebensolche Benchmarkings etabliert werden, um vergleichbare Datenlagen etablieren zu können. Ein weiteres Problem ist, dass diese Ansätze häufig sehr anfällig für Adversarial Example Angriffe sind, die in \ref{ML-Evasion} genauer betrachtet werden.