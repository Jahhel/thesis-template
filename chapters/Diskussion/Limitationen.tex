\section{Limitationen}
Aufgrund der eingeschränkten Möglichkeiten innerhalb einer Bachelorarbeit konnten selbstverständlich nicht alle Probleme und Eventualitäten in der Untersuchung berücksichtigt werden, die folgende Forschung in Betracht ziehen sollte.
So sind neben der kleinen Stichprobe von drei Durchläufen pro Malware-AV Scanner Paar auch die eingeschränkte Auswahl an Obfuskatoren/Packern zu bemängeln. Des Weiteren wurde zwar die Ausführbarkeit von Malware mittels der AV Scanner überprüft, allerdings kann keine Garantie dafür übernommen werden, dass die Malware am Ende der Obfuskation noch immer ihren ursprünglichen Zweck erfüllen kann. Dies ist besonders relevant da Holm et al \cite{holm_2023_hide} einen signifikanten Unterschied in dem Anteil der interaktiven Malware fanden, welche von AV-Scannern nicht erkannt wurden und welche einen tatsächlichen Shell-Zugriff ermöglichten.
Die Auswahl der AV Scanner war ebenso von dem geprägt, was innerhalb der SAP am häufigsten vorkommt und an Tools bereits vorhanden und Einsatzfähig war. Eine Erweiterung der AV-Scanner oder eine kompetetive Erweiterung von einer Scan-Batterie gegen die Obfuskationsgenerierung dieses Algorithmusses wäre sehr wünschenswert.