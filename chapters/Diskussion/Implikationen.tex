\section{Implikationen}
\subsection{Red Teaming}
Die Verwendung des vorgestellten Tools zeigt bei den verwendeten Testcases unter ähnlichen Bedingungen, wie sie in Red Teaming Exercises vorherrschen, Erfolg innerhalb der erwünschten Parameter. Damit lässt sich dieses Tool selbst als Erfolg verbuchen und ermöglicht somit eine neue Perspektive in der Nachstellung von Multipacked Malwareangriffen auf Firmensysteme. 
\subsection{AV Hersteller}
Wie \cite{murali_2023_evolving} bereits angestoßen, könnten Malwarevarianten zum Training  von AV Software verwendet werden und so als 'Antigen' für das 'Immunsystem' bzw. Malwaredetektion dienen. Andererseits hat\cite{dyrmishi_2023_on} festgestellt, dass dieser Ansatz für PE Files wenig erfolgsversprechend ist und nur in speziellen Anwendungsfällen, wie Botnetdetektion einen Mehrwert verspricht. Einschränkend lässt sich dazu allerdings sagen, dass in diesen Untersuchungen Angriffe auf 'unrealistischen' Adversarial Attacks basierten.
\subsection{Genetische Algorithmen in der IT Sec}
Wie schon in \ref{chapter:theorie} angerissen, sind auch heute noch Genetische Algorithmen ein Tool der Wahl für Optimierungsprobleme im Cyber-Security Bereich. Mit dem Ansatz, dass Genetische Algorithmen zur Komposition und Kombination von verschiedenen Angriffswegen als Framework verwendet werden kann, wurde ein weiterer Anwendungsfall belegt und ermöglicht somit neue Forschungen und verbesserte Ergebnisse in der Praxis.