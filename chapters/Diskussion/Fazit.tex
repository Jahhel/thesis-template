\section{Fazit}
Hinsichtlich der Fragestellung inwieweit Obfuskatorkaskaden sich durch genetische Optimierung für Red Teaming Assignments verwenden lassen, hat sich durch die Hypothesentests ein mehrheitlich positives Bild gezeigt. So ergab sich, dass es signifikante Unterschiede zwischen verschiedenen Malwarearten gab und der Anzahl an Obfuskationen, die es benötigte, bis sie von AV Scannern nicht mehr erkannt wurden, während die PE Korrumpierung bei mehrstufiger Obfuskierung zu keinem größeren Problem wurde.
Außerdem war es möglich die benötigte Zeit und eine solide Erfolgsrate mithilfe des Tools zu erzielen.

Für die Forschung bedeutet dies, dass eine Untersuchung von mehrstufiger Obfuskierung, ihren Auswirkungen auf die Funktionsweise und Detektion durch Malware nötig erscheint, gerade wenn man sie in Verbindung mit den moderneren und Zeitaufwändigeren Verfahren für Adversarial Attacks kombiniert.

Red Teams haben nun eine Möglichkeit in kurzer Zeit eine geringe Auswahl an Malwarevarianten zu generieren, welche von AV Software nicht erfasst wird und die konfigurierbar ist. Damit entfällt eine Menge an manueller Arbeit, die anderweitig eingesetzt werden kann.

AV Hersteller stehen vor der Herausforderung sich immer neuen und komplexeren Bedrohungen gegenüber zu sehen und gegen diese Verteidigen zu müssen. Gerade für Enterprisekunden kann die Gefahr durch die hier untersuchte mehrfach obfuskierte Malware aber relevant sein, sodass sich auch diese dem genannten Phänomen annehmen sollten.