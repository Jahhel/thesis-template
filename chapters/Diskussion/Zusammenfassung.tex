\section{Zusammenfassung}
Das Experiment zur Optimierung von Obfuskatorkaskaden für Metapretermalware hat ergeben, dass der vorgestellte Algorithmus innerhalb von 10 Minuten oder weniger für jeden verwendeten Malwaretyp und AV Scanner eine Lösung zu konstruieren, die vom AV Scanner nicht erkannt wurde. Ein großer Anteil dieser Lösung beinhaltet dabei mehr als einen Obfuskationsschritt, was besonders deutlich am Beispiel der Reverse Shell wurde, bei der keine einzige Obfuskation gelungen ist, welche nur einen Schritt beinhaltete. Des weiteren hat sich gezeigt, dass der Algorithmus nur einen geringen Anteil an mehrstufigen Lösungen generiert hat, welche nicht ausführbar waren. Damit lässt sich konkludieren, dass der vorgestellte Algorithmus die Anforderungen für Red Team Assignments erfüllt und für den vorgesehenen Einsatz verwendbar ist. Weiter hat sich gezeigt, dass die Erweiterung des Möglichkeitenraums auf mehrstufige Obfuskatioinsketten eine akzeptable Laufzeit bei erhöhter Evasionswahrscheinlichkeit ermöglicht hat.
