%*******************************************************
% Abstract in German
%*******************************************************
\begin{otherlanguage}{ngerman}
	\pdfbookmark[0]{Zusammenfassung}{Zusammenfassung}
	\chapter*{Zusammenfassung}
    Diese Studie untersucht den Mehrwert, den mehrstufige Obfuskation von Malware zum Entgehen von Detektion durch Antivirussoftware für Red Teaming Einsätze hat. Zur Generation dieser Obfuskationen wurde ein genetischer Algorithmus verwendet, der die Konfigurationen erstellt und prüft. Entgegen den Erwartungen zeigt sich durch das Stapeln von Obfuskationstechniken keine große Korrumpierung der PE -Malware Files. Es hat sich allerdings gezeigt, dass die Anzahl der nötigen Obfuskationsschritte in Zusammenhang mit der Art der Malware stehen, welche obfuskiert werden soll. So benötigten in der Untersuchung (N=18) Reverse Shell Malware mindestens zwei Obfuskationsschritte, damit sie nicht von der Antiviren-Software erkannt werden konnte. 
    Des Weiteren war es möglich, eine Erfolgsquote von mehr als 75\% des Tools nachzuweisen und eine Laufzeit von unter 10 Minuten beizubehalten, was für die Verwendung in simulierten Cyberangriffen essentiell ist, um einen Mehrwert bieten zu können.
    Stichpunkte: Genetische Algorithmen, Obfuskator, Malware, Malware Detektion, Malware Detektion Evasion, Windows Portable Executable File, IT Sicherheit, Red Team, Simulated Cyber Attack
\end{otherlanguage}
