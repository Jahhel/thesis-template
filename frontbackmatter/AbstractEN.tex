%*******************************************************
% Abstract in English
%*******************************************************
\pdfbookmark[0]{Abstract}{Abstract}


\begin{otherlanguage}{american}
	\chapter*{Abstract}
    This study investigates the value in multi stepped obfuscation of Windows-PE-File Malware generated by Metasploit. For generating evasive configurations and malware variants a genetic algorithm was designed. Contrary to expectations the percentage of corrupted multi-obfuscated samples was rather low. Instead it could be proven that the number of neccessary steps for a succesfull evasions are depened upon the malware type, as especially the interactive reverse shell required at least 2 Obfuscation steps, before it was not detected. 
    In the study 17 out of 18 executions of the tool in its unoptimized state lead to satisfactory results, while the unsucessful run could be repeated and then produced 4 evasive samples. These findings stand while the tool does adhere to the timeconstraint for run time of 10 minutes and other red teaming constraints that advocate for its usability.
\end{otherlanguage}
